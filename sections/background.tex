\section{Background} \label{background}

% Problem definition: what is the problem?
This project fits within the ``Vanishing and Appearing Sources during a Century of Observations'' (\vasco) initiative (see \url{https://vasconsite.wordpress.com}), aiming for finding inexplicable effects between all-sky surveys. The \vasco project is a collaboration between astronomers and information technology researchers, and incorporates explicitly a component of citizen science.

This master project will focus on the Machine Learning (ML) component. This component is situated in-between the developed graphical user interface (see \url{http://user.it.uu.se/~kripe367/MLblink/}), and the computational infrastructure building on cloud-computing infrastructure (see \todo{URL missing}).

% What solution does the thesis propose?
The precise objective is to develop and test a code that serves up the most interesting candidates that a ML-based solution can mine from the historically observations sky surveys. The ML component is described as ML-Blink, and is based on methods of active and online semi-supervised learning.

The huge data to be mined is collected in efficient (SQL-based) databases stored on mentioned cloud service. The overall idea is then to develop a code that can {\em crawl} these databases for interesting candidates. Crawling here points to effective code (a ``roombai'') that mines the databases for a given matching criteria. This matching criteria is given by ML-Blink, and is going to change (``learned'') in time. The present applications translates those generic ideas in terms of astronomical images.