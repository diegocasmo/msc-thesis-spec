\section{Project description} \label{description}

This section goes into more detail on how the problem described in \ref{background} will be addressed. The problem is further split into different tasks which specify the distinct phases of the project. 

\subsection{Establish feasibility of the AI-Crawler}

For a fixed, small-scale dataset, make a code establishing feasibility of the AI-Crawler. This example is to be implemented locally.

\subsection{Connecting the GUI ML-Blink to the database}

Implementing a scheme that allows the graphical user interface (GUI) ML-Blink to send information back to the database. This means that the ML-Blink GUI needs a persistence layer (RESTful API), a development effort that uses existing frameworks coded in JSON.  This will allow us:

\begin{enumerate}[(a)]
  \item for different users to login under their credentials,
  \item to store and exchange comments of users on a specific case,
  \item to store user-specific statistics, and
  \item to send information of findings to a central database.
\end{enumerate}

\subsection{Scaling up to the complete datasets}

Scaling up the solution to the complete datasets. Since the data is served by different international computing infrastructures, scaling up should be aware of issues of caching. 
 
\subsection{Thesis}

Testing the solution and writing the subsequent report. The final report should summarize findings, describe a case study, and identify shortcomings of the developed system.