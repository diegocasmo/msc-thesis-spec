\section{Approach}

This section outlines tools and methods which will be used to solve each of the tasks described in section \ref{description}.

\subsection{Methods}
\begin{itemize}
    \item Literature study
    \item The persistence layer will consist of a RESTful API which will communicate using the JSON notation
    \item A cloud environment might be used for deployment
\end{itemize}

\subsection{Tools}
 
\begin{itemize}
    \item The \Python programming language (\url{https://www.python.org/}) will be used to implement the persistence layer
    \item The Conda package management system (\url{https://conda.io/docs/}) might be used to manage \Python dependencies
    \item The \Python library \scikit (\url{https://scikit-learn.org/stable/}) will be used to develop the AI-Crawler
    \item The \Python library \ImageDict (\url{https://github.com/kobejohn/imagedict}) might be used for smart caching of images
    \item The thesis report will be written in \LaTeX ~via Overleaf (\url{https://www.overleaf.com})
    \item Git as the version-control system
\end{itemize}

\subsection{Relevant literature}

This section discusses relevant literature which will be complemented by more as the project is carried out.

Recommender systems are studied in detail in \cite{BOOK:RST}. ML-Blink is a recommender system, since it attempts to serve up interesting candidates given a matching criteria. This implies the ``goodness'' of such candidates must be measured. The discussion of recommender systems provided in \cite{BOOK:RST} will serve as a foundation to better understand the goals of recommender systems, existing models, as well as online and offline evaluation techniques.

The theoretical foundations of machine learning for a similar problem are examined in \cite{article:fastlund}. These concepts will likely be useful when conducting the aforementioned project.

Finally, a study of the most important statistical learning methods is provided in \cite{BOOK:ESL}. This is a well known book in the field of statistical learning, and it examines topics of importance related to this project such as supervised and unsupervised learning.

\subsection{Relevant courses}
Table \ref{approach:relevant_courses} shows a list of courses taken during the masters degree studies which might be helpful while conducting this thesis project.

\begin{table}[ht]
  \centering
  \begin{tabularx}{\textwidth}{|l|X|}
    \hline
    Code & Course name \\
    \hline
    1DL360 & Data Mining I \\
    1MD032 & Intelligent Interactive Systems \\
    1DL340 & Artificial Intelligence \\
    1DT071 & Machine Learning \\
    1DL231 & Algorithms and Data Structures II \\
    1DL481 & Algorithms and Data Structures III \\
    1DL400 & Database Design II \\
    1MD016 & Human-Computer Interaction \\
    1DL610 & Software Testing \\
    \hline
  \end{tabularx}
  \caption{List of courses taken during the masters degree studies at Uppsala University that could be relevant to conduct the project.}
  \label{approach:relevant_courses}
\end{table}